\documentclass{article}

\usepackage[utf8]{inputenc}
\usepackage{xspace}

\newcommand{\fp}{\ensuremath{\mathcal{FP}}}
\newcommand{\sharpp}{\ensuremath{\# \mathcal{P}}}
\newcommand{\cc}{\texttt{C++}\xspace}
\newcommand{\irram}{\texttt{iRRAM}\xspace}

\begin{document}

	Analytic functions play a central role in analysis and have therefore also been investigated in Computable Analysis and Real Complexity Theory.
	Many operators that are hard in the general case map polynomial time computable functions to polynomial time computable functions when restricted to analytic functions.
	It can for example be shown, that the anti-derivative of every polynomial time computable function is polynomial time computable if and only if $\fp = \sharpp$ (Friedman).
	The anti-derivative of a polynomial time computable analytic function, however, can be computed in polynomial time without further assumptions.

	Analytic functions are represented by power series.
	However, it is known that the evaluation of power series is not uniformly computable.
	Therefore it is necessary to include additional information about the function in the representation.
	A representation that makes many operators uniformly polynomial time computable is extending the series by two Integers that fulfill certain conditions.

	Norbert Müller's \cc Framework \irram provides data types for error-free computations with real numbers \cite{Mueller00}.
	A prototype for a data type for representing analytic functions in the above sense was implemented by Florian Steinberg.
	The talk focuses on the task of implementing Analytic Continuation (i.e. expanding the domain of an analytic function) in \irram using this data type.
	For this it is necessary to compute the Taylor-coefficients for a given function.

\end{document}