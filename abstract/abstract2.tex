\documentclass{article}
\usepackage{a4}
\usepackage[utf8]{inputenc}
\usepackage{xspace}
\usepackage{amsmath,amssymb}
\newcommand{\DD}{\mathbb D}
\newcommand{\RR}{\mathbb R}
\DeclareMathOperator{\NN}{\mathbb N}
\DeclareMathOperator{\C}{\mathcal C}
\DeclareMathOperator{\laplace}{\Delta}
\bibliographystyle{amsalpha}

\newcommand{\p}{\ensuremath{\mathcal P}\xspace}
\newcommand{\np}{\ensuremath{\mathcal{NP}}\xspace}
\newcommand{\fp}{\ensuremath{\mathcal{FP}}\xspace}
\newcommand{\sharpp}{\ensuremath{\# \mathcal{P}}\xspace}
\newcommand{\cc}{\texttt{C++}\xspace}
\newcommand{\irram}{\texttt{iRRAM}\xspace}
\newcommand{\REAL}{\texttt{REAL}\xspace}

\begin{document}
\title{Analytic Functions in \irram\thanks{%
Supported in part by \emph{JSPS Kakenhi} projects
\texttt{23700009} and \texttt{24106002},
by \emph{7th EU IRSES} project \texttt{294962},
by \emph{DFG} project \texttt{Zi\,1009/4-1},
and by \emph{DAAD}.}}
\author{Akitoshi Kawamura$^1$, \quad Florian Steinberg$^2$, \quad Holger Thies$^{1,2}$ 
\\
$^1$ The University of Tokyo (JAPAN), \quad $^2$ TU Darmstadt (GERMANY)}
\date{}
\maketitle
\noindent
The \emph{Type-2 Theory of Effectivity} (TTE) provides a sound framework for
investigating the computability of real numbers, sequences,
functions, and operators. 
While a power series with uniformly computable coefficient sequence 
does give rise to a computable function on every fixed compact
subset of the disc of convergence, the underlying algorithm must
in addition to said sequence `know' certain additional 
(integer) parameters of it \cite{Mueller95}.
Real Computability Theory thus exhibits a canonical mixed real/integer 
representation of power series as interface declaration of practical 
implementations. In fact the \irram library conveniently realizes an
abstract data type \REAL for imperative programming in \cc,
internally as intervals of arbitrary precision
but appearing to the user as primitive with sound
(namely multivalued) overloaded semantics for tests/branches \cite{Mueller00}.
Moreover, Real Complexity Theory shows
that the usual operations on (real) analytic functions can
be computed uniformly within time polynomial in the binary
output precision $n$ plus the aforementioned integer parameters \cite{Kawamura2012}.
This is as opposed to smooth (=infinitely often differentiable)
functions where maximization has been shown to correspond to
the \p--vs.--\np  Millennium Prize Problem 
and integration even to \fp--vs--\sharpp \cite{MR666209,MR748898,MR1137517}.

We present a prototype implementation of power series 
and analytic functions on a fixed rectangular domain 
as abstract `arrow' data type in \irram,
supporting basic operations like evaluation,
point wise addition and multiplication, composition,
differentiation and integration. Building on that,
we explore the practical potential for analytic 
continuation by iterated evaluation/interpolation
in exact real arithmetic. Our empirical evaluation
compares practical with predicted running times and suggests
a refinement to the above parametrized complexity analysis.

\small

\bibliography{bib}{}
\end{document}
