\documentclass{article}

\usepackage[utf8]{inputenc}
\usepackage{xspace}
\usepackage{amsmath,amssymb}
\newcommand{\DD}{\mathbb D}
\newcommand{\RR}{\mathbb R}
\DeclareMathOperator{\NN}{\mathbb N}
\DeclareMathOperator{\C}{\mathcal C}
\DeclareMathOperator{\laplace}{\Delta}
\bibliographystyle{amsalpha}

\newcommand{\p}{\ensuremath{\mathcal P}\xspace}
\newcommand{\np}{\ensuremath{\mathcal{NP}}\xspace}
\newcommand{\fp}{\ensuremath{\mathcal{FP}}\xspace}
\newcommand{\sharpp}{\ensuremath{\# \mathcal{P}}\xspace}
\newcommand{\cc}{\texttt{C++}\xspace}
\newcommand{\irram}{\texttt{iRRAM}\xspace}

\begin{document}
\section*{Analytic Functions in \irram}
	Real Computability Theory originates from Turing's famous 1936 paper \cite{zbMATH03025275}, was developed further in the 50's \cite{MR0089809,MR0105357} and is now an established foundation for algorithmic considerations about continuous structures.
	Real Complexity Theory was devised and developed by Friedmann and Ko \cite{MR666209,MR1137517} and has widely been accepted as the correct notion of feasibility \cite{Weihrauch}.

	It turned out that within this framework many basic operators on real functions preserve polynomial time computability if and only if inclusions of discrete complexity classes are non strict.
	More explicitly: parametric maximization corresponds to $\p$ vs. $\np$ \cite{MR666209} and integration to the stronger $\fp$ vs. $\sharpp$ \cite{MR748898}.
	These results basically say that it is probably very difficult to find algorithms to carry out these basic operations on functions.
	The same remains true if one restricts to smooth functions.
	However, the situation improves drastically if only analytic functions are considered: Many operators that are hard in the general case become computable in polynomial time \cite{Kawamura2012}.

	Analytic functions are represented by power series.
	However, it is known that not even the evaluation of power series is uniformly computable.
	A useful representation for analytic functions therefore has to include some additional information.
	A representation that makes many operators uniformly polynomial time computable is extending the series by two Integers that fulfill certain conditions.

	Norbert Müller's \cc library \irram provides an implementation of Real Complexity Theory and makes it accessible for numerical scientists \cite{Mueller00}.
	A prototype of a data type representing analytic functions as power series in the above sense has been implemented by Florian Steinberg. 
	In this talk we show how analytic functions can be implemented and used in \irram. 
	An application we focus on is analytic continuation, i.e. expanding the domain of an analytic function. The domain is extended by computing a new Taylor series for the function around another point in the domain. An algorithm for computing Taylor series for a given analytic function was proposed by Norbert M\"uller \cite{Mueller95}. The data type was extended by this algorithm.  

\bibliography{bib}{}
\end{document}